\documentclass{article}

\usepackage{geometry}

\setlength{\parindent}{0pt}
\setlength{\parskip}{5pt}

\title{CORE111 Logical Problem Solving\\Homework 3}
\author{Mir Mehdi Ali Baqri}

\begin{document}
\maketitle

\section{Proofs}

(a) \textbf{Premise:} \\
\begin{enumerate}
   
    \item $P \rightarrow  \neg M$ 
    \item $\neg M \rightarrow \neg C$ 
    \item $\neg C \rightarrow \neg L$ 
    \item $\neg P \rightarrow \neg E$ 
    \item $\neg E \rightarrow \neg C$ 
    \item $P \vee \neg P$ 
    \end{enumerate}

\textbf{Conclusion:}
$ \neg L $ \\ 
\\
\textbf{Proof:} \\
By applying Hypothetical syllogism on (3) and (5) we get: \\
$\neg E \rightarrow \neg L       - 7 \\$
\\
Now by applying Hypothetical syllogism on (4) and (7) we get: \\ 
$\neg P \rightarrow \neg L      - 8 \\$
\\
Now using 8 and 6 we get (Modus Ponens): \\
$\neg P \rightarrow \neg L \\ $
$\neg P $ \\
$ \noindent\rule{1.9cm}{0.4pt}\\ $
$\neg L  \\ $
\\
Now by applying Hypothetical syllogism on (1) and (2) we get: \\
$ P \rightarrow \neg C - 9 \\ $
\\
By using hypothetical syllogism on (9) and (3): \\
$ P \rightarrow \neg L $ - 10 \\
\\
Now by using (10) and (6) \\
$ P \rightarrow \neg L $ \\
$P \vee \neg P$ \\
\noindent\rule{1.9cm}{0.4pt}\\
$ \neg L$ \\
\\
(b) \textbf{Premise:} 
\begin{enumerate}
    \item $\neg C \rightarrow C \vee (J \rightarrow D)$
    \item $C \rightarrow (C \wedge U)$
    \item $ \neg (C \wedge U)$
    \item $\neg D$ \\
    \end{enumerate}
\textbf{Conclusion:}
$ \neg J $ \\
\\
Applying Modus Tollens on (2) and (3) we get: \\
$ \neg C $ - 5 \\
\\
Now applying Modus ponens on (5) and (1) we get: \\
$\neg C \rightarrow C \vee (J \rightarrow D)$ \\
$ \neg C $ \\
\noindent\rule{1.9cm}{0.4pt}\\
$ C \vee (J \rightarrow D)$ - 6 \\
\\
Finally, by using (6) , (5) and (4) we can condlude: \\
$ C \vee (J \rightarrow D)$ \\
$ \neg C $ \\
$\neg D$ \\
\noindent\rule{1.9cm}{0.4pt}\\
$ \neg J $ \\
\\
(c)  \textbf{Premise:} 
\begin{enumerate}
    \item $ M \vee P $
    \item $(P \vee S) \rightarrow (R \wedge D) $
    \item $\neg M $
\end{enumerate} 

\textbf{Conclusion:}
$ R $ \\
\\
By applying disjunctive syllogism on (1) and (3) : \\
$ M \vee P $ \\
$\neg M $ \\
\noindent\rule{1.9cm}{0.4pt}\\
$ P $ - 4 \\
\\
By using (4) and (2) we can conclude that:
$(P \vee S) \rightarrow (R \wedge D) $ \\
$ P $ \\
\noindent\rule{1.9cm}{0.4pt}\\
$ (R \wedge D) $ - 5 \\
\\
Now by Simplification of (5) we get:
$ (R \wedge D) $ - 5 \\
\noindent\rule{1.9cm}{0.4pt}\\
$ R $ \\
\\
(d) \textbf{Premises:}
 \begin{enumerate}
    \item $\neg E \vee (B \wedge P)$
    \item $ \neg E \vee (G \wedge W)$
    \item $\neg P \vee \neg W$ 
    \end{enumerate}
\textbf{Conclusion:}
$\neg E $ \\
\\
By applying resolution on (3) and (2) we get: \\
$\neg P \vee \neg W$ \\
$\neg E \vee (G \wedge W)$ \\
\noindent\rule{1.9cm}{0.4pt}\\
 $ \neg P \wedge \neg E $ - 4 \\
 \\
 By applying resolution again on (4) and (1) we get: \\
 $ \neg P \wedge \neg E $ \\
 $\neg E \vee (B \wedge P)$ \\
 \noindent\rule{1.9cm}{0.4pt}\\
 $\neg E $\\
\\

(e) \textbf{Premise:} \\
\begin{enumerate}
    \item $G \rightarrow A$
    \item $G \rightarrow L$
    \end{enumerate}
\textbf{Conclusion:} 
$G \rightarrow (A \wedge L)$ \\
\\
Changing implication into OR : \\
\begin{enumerate}
    \item $\neg G \vee A$
    \item $\neg G \vee L$
    \end{enumerate}
We can also write the premise together as: 
 $(\neg G \vee A) \wedge (\neg G \vee L)$\\
 \\
 Taking $ \neg G $ common : \\
 $ \neg G \vee (A \wedge L)$\\
 \\
 This can also be written as : \\
 $G \rightarrow (A \wedge L)$ \\
 \\
(f) \textbf{Premise:} \\
\begin{enumerate}
    \item $H \rightarrow D$
    \item $U \rightarrow S$
    \end{enumerate}
    \textbf{Conclusion:}
    $(H \wedge U) \rightarrow (S \wedge D)$ \\
By using (1) and ACP :\\
 $H \rightarrow D$ \\
 $H$ \\
 \noindent\rule{1.9cm}{0.4pt}\\
 $ D $ \\
 \\
 By using (2) and ACP: \\
 $U \rightarrow S$ \\
 $U$ \\
 \noindent\rule{1.9cm}{0.4pt}\\
 $S$ \\
 \\
 Now assuming "H" and "U" to be true we can say that the required conclusion i.e. $(H \wedge U) \rightarrow (S \wedge D)$ is true.
 

(g) \textbf{Premise:} \\
\begin{enumerate}
\item $(Z \rightarrow C) \rightarrow B$
\item $(V \rightarrow Z) \rightarrow B$
\end{enumerate}

\textbf{Conclusion:} 
$B$ \\
\\
Changing the implication of (1) into OR : \\
$\neg(\neg Z \vee C) \vee B$ \\
\\
By De Morgan's theorem : \\
$(\neg \neg Z \wedge C)  \vee B$ \\
\\
This is equal to: \\
$Z \vee B$ - 3\\
\\
Now, changing the implication of (2) into OR : \\
$\neg (\neg V \vee Z) \vee B$\\
\\
By De Morgan's theorem : \\
$ (\neg \neg V \wedge Z) \vee B$\\
\\
This is equal to:
$ \neg Z \vee B$ - 4\\
\\
By using (3) and (4) :\\
$Z \vee B$ \\
$ \neg Z \vee B$ \\
\noindent\rule{1.9cm}{0.4pt}\\
$ B $



\section{More Proofs}

(a) \textbf{Premise:} \\
\begin{enumerate}
    \item $S \rightarrow V $\\
    \end{enumerate} 
\textbf{Conclusion:}
$ \neg V \rightarrow \neg S $ \\
\\
The conclusion can be achieved if we see the contra-positive of the premise i.e : \\
$ \neg V \rightarrow \neg S $ \\
\\
(b)\textbf{Premise:}
\begin{enumerate} 
\item $(\neg S \wedge V) \rightarrow \neg P$
\item $P$
\item $V$
\end{enumerate}
\textbf{Conclusion:}
$ S $ \\
\\
By applying Modus Tollens on the contra-positive of (1) and on (2) we get: \\
$ P \rightarrow \neg (\neg S \wedge V) $ \\
$ P $ \\
\noindent\rule{1.9cm}{0.4pt}\\
$\neg (\neg S \wedge V) $ - 4 \\
\\
By De Morgan's theorem (4) is equal to : \\
$ S \vee \neg V $ \\
\\
Now by using (4) and (3) : \\
$ S \vee \neg V $ \\
$V$ \\
\noindent\rule{1.9cm}{0.4pt}\\
$ S $ \\
\\
(c) \textbf{Premise:}
\begin{enumerate}
\item $\neg S \rightarrow (\neg P \vee \neg V)$
\item $P \wedge V$
\end{enumerate}
\textbf{Conclusion:}
$ S $ \\
\\
By De Morgan's Theorem (1) can be written as: \\
$\neg S \rightarrow \neg ( P \wedge  V)$ \\
\\
Now by applying Modus Tollens on (1) and (2) : \\
$\neg S \rightarrow \neg ( P \wedge  V)$ \\
$P \wedge V$ \\
\noindent\rule{1.9cm}{0.4pt}\\
$\neg \neg S $ -5 \\
\\
(5) can also be written as : $S$

\end{document}